\documentclass{article}
\usepackage{../fasy-hw}
\usepackage{ wasysym }

%% UPDATE these variables:
\renewcommand{\hwnum}{2}
\author{Peter Gifford, Ren Wall, Kyle Brekke, Madison Hanson Group: 7}
\collab{}
\date{due: 20 September 2019}

\begin{document}

\nextprob
Give a linear-time algorithm that takes two sorted arrays of real numbers as
input, and returns a merged list of sorted numbers.  You should give your answer
in pseudocode.    Your answer should contain:
\begin{itemize}
    \item A prose explanation of the algorithm.
    \item Psuedocode. (Be sure to review the two resources on pseudocode that were
        posted as readings for Week 2!  I also suggest the algorithm /
        algorithmx package in LaTex.)
        
         \begin{algorithm}
        \caption{Merged list of sorted numbers from two sorted lists}\label{sorted list}
        \begin{algorithmic}[1]
        \Procedure{Merge}{$A,B$}\Comment{A and B are sorted lists this sorts them into list c}
        		\State \textbf{in:}  Sorted lists A,B
		\State \textbf{out:}  Sorted list c, the combination of A and B
        		\State $c\gets list;$
		\State $i, j\gets 0;$
		\While{$i < A.size()  \&\&  j < B.size()$}
			\If{$A.get(i) == B.get(j)$}
				\State c.add(A.get(i), B.get(j));
				\State i++;
				\State j++;
			\ElsIf{$A.get(i) < B.get(j)$}
				\State c.add(A.get(i));
				\State i++;
			\ElsIf{$A.get(i) > B.get(j)$}
				\State c.add(B.get(j))
				\State j++;
			\EndIf
		\EndWhile
		\If{$i == A.size()$}
			\While{$j < B.size()$}
				\State c.add(B.get(j));
				\State j++;
			\EndWhile
		\EndIf
		\If{$j == B.size()$}
			\While{$i < A.size()$}
				\State c.add(A.get(i));
				\State i++;
			\EndWhile
		\EndIf
		\State return c;
	\EndProcedure
	\end{algorithmic}
	\end{algorithm}
        
    \item The decrementing function for any loop or recursion.
    
        { D(x) : x = ( length(A) + length(B) )-(i + j) } $\rightarrow \mathbb{Z} \newline$
    Loop terminates when D reaches 0. \newline
    { D(x) : x = ( length(A) - i ) } $\rightarrow \mathbb{Z} \newline$
     Loop terminates when D reaches 0. \newline
    { D(x) : x = ( length(B) - j ) } $\rightarrow \mathbb{Z} \newline$
     Loop terminates when D reaches 0. \newline
    
    \item Justification of why the runtime is linear.
    
        The algorithm will go through every item in the lists exactly once since the counters i and j will increment until they hit the array size and therefore no item in the lists will be have more than O(1) spent on it. This mean that O(A.size()+B.size()) is the complexity and therefore it is run in linear time.
    
\end{itemize}

\nextprob
EPI 15.4 (Generate the Power Set) gives code to compute the power set of a set
(without duplicates).  Present this problem and solution in your own words using
pseudocode.

  	\begin{algorithm}
	\caption{Power Sets}\label{power sets}
        \begin{algorithmic}[1]
		 \Procedure{generatePowerSets}{$inputSet$}\Comment{Startup function}
		 \State \textbf{in:} List of integers that is a the inputSet
		 \State \textbf{out: } List of Lists of Integers that are the power sets;  
		 \State powerSet $\gets$ list;
		 \State newList $\gets$ list;
		 \State directedPowerSet(inputSet, 0, newList, powerSet);
		 \State return powerSet;
		 \EndProcedure
		 
		 \Procedure{directedPowerSet}{$inputSet, toBeSelected, selectedSoFar, powerSet$}
		 	\State \textbf{in:} inputSet : the original input set, toBeSelected : the spot in inputSet that the algorithm is checking, selectedSoFar : list of spots in inputSet already checked, powerSet : list of power sets already selected
			\State \textbf{out: } None
		 	\If{$toBeSelected == inputSet.size()$}
				\State powerSet.add(selectedSoFar.asList());\Comment{Adds all of selected so far because they represent a powerSet to powerSet and ends because there is nothing left to check}
				\State return;
			\EndIf
			\State selectedSoFar.add(inputSet.get(toBeSelected)); 
			\State directedPowerSet(inputSet, toBeSelected + 1, selectedSoFar, powerSet);
			\State selectedSoFar.remove(selectedSoFar.size() - 1);
			\State directedPowerSet(inputSet, toBeSelected + 1, selectedSoFar, powerSet);
		 \EndProcedure
	\end{algorithmic}
	\end{algorithm}


\nextprob
In EPI 15.1 (The Towers of Hanoi Problem), prove that the algorithm as presented
terminates.  In particular, you should give the decrementing function for the
recursion.

\nextprob
For the stock market problem discussed in class on September 6th (and in CLRS
4.1), walk through
the algorithm for the following input:
$$\mathtt{price} = \{ 3, 6, 8, 2, 1, 10, 5, 7 \}. $$

BuySell(\{n[1]:3, n[2]:6, n[3]:8, n[4]:2\}), BuySell(\{n[5]:1, n[6]:10, n[7]:5, n[8]:7\})\newline
BuySell(\{n[1]:3,n[2]: 6\}) BuySell(\{n[3]:8, n[4]:2\}), BuySell(\{n[5]:1, n[6]:10\}), BuySell(\{n[7]:5, n[8]:7\})\newline
compare(\{n[1]:3,n[2]:6\},\{n[3]:2,n[4]:8\},\{n[1]:3,n[4]:8\}) = \{n[3]:2,n[4]:8\}, compare(\{n[5]:1,n[6]:10\},\{n[7]:5, n[8]:7\},\{n[5]:1,n[8]:7\}) = \{n[5]:1, n[6]:10\}\newline
compare(\{n[3]:2,n[4]:8\},\{n[5]:1,n[6]:10\},\{n[3]:2,n[6]:10\}) = \{n[5]:1,n[6]:10\}\newline
result = \{n[5],n[6]\}

\nextprob
Prove using induction that the closed form of:
$$T(n) = \begin{cases}
            1        & n=1\\
            T(n-1)+n & n>1
         \end{cases}
$$
is $O(n^2)$.

\nextprob
What is the closed form of the following recurrence relations?  Use Master's
theorem to justify your answers:
\begin{enumerate}
    \item $T(n) = 16 T(n/4) + \Theta(n)$
    
    $a = 16, b = 4, n^2, f(n) = n, case  1\newline$
    $\epsilon = 1$
    T(n) =  $\Theta(n^2)$
    
    \item $T(n) = 2 T(n/2) + n \log{n}$
    
    $a = 2, b = 2, n^1, f(n) = n\log{n}, case 3\newline$
    $f(n) = \Theta(n^c), c = 2\newline$
    $\log_{2}2 < 2$ satisfies condition for case 3 \newline
    T(n) = $\Theta(n\log{n})$
    
    \item $T(n) = 6 T(n/3) + n^2 \log{n}$
    
   $a = 6, b = 3, n^1.6, f(n) = n^2, case 3\newline$
   $f(n) = \Theta(n^c), c = 2\newline$
    $\log_{3}6 < 2$ satisfies condition for case 3 \newline
    T(n) = $\Theta(n^2)$
    
    \item $T(n) = 4 T(n/2) + n^2$
    
    $a = 4, b = 2, n^2, f(n) = n^2, case 2 \newline$
    T(n) = $\Theta(n^2\log{n})$
    
    \item $T(n) = 9 T(n/3) + n$
    
    $a = 9, b = 3, n^2, f(n) = n, case 1\newline$
    $\epsilon = 1$
    T(n) = $\Theta(n^2)$
    
    
\end{enumerate}
Note: we assume that $T(1)=\Theta(1)$ whenever it is not explicitly given.

\nextprob
\emph{The skyline problem:} You are waiting for the ferry across the river to
get into a big city, and notice
$n$ buildings in front of you.  You take a photo, and notice that each building
has the silhouette of a rectangle.  Suppose you  represent each building as a
triple $(x_1,x_2,y)$, where the building can be seen from $x_1$ to $x_2$
horizontally and has a height of $y$.  Let $\mathtt{rect(b)}$ be the set of
points inside this rectangle (including the boundary).  Let $\mathtt{building}$ be the set of $n$
triples. Design an algorithm that takes $\mathtt{buildings}$ as input, and
returns the skyline, where the skyline is a sequence of $(x,y)$ coordinates
defining $\cup_{b \in \mathtt{buildings}} \mathtt{rect}(b)$.

\nextprob
The \texttt{rand()} function in the standard C library returns a
uniformly random number in \texttt{[0,RANDMAX-1]}. Does \texttt{rand}()$\mod n$
generate a number uniformly distributed in $[0,n-1]$?

Note I: This is the second variant in EPI 5.12.

Note II: When asked questions of this form, you are expected to justify your
answer.

\nextprob

Algorithms where we use randomization to find a deterministic answer are known
as Las Vegas algorithms.  Monte Carlo algorithms also use randomization, but
might not always give the right answer; however, they either have a high
probability of being correct or close to correct.

\begin{enumerate}[(a)]
    \item Give a Monte Carol algorithm to estimate~$\pi$.
    \item Let $n$ be the number of random numbers used by your algorithm.
        Explain why as $n \to \infty$, the expectation of the output for your
        algorithm is~$\pi$.
    \item Implement this algorithm and plot a line graph of
        the values returned for at least $10$ values of~$n$.
\end{enumerate}

Note: We can use the function \texttt{randReal}$(a,b)$ that returns a random
real number between $a$ and $b$ inclusive.

\end{document}
